%% ------------------------------------------------------------------------- %%
\setcounter{chapter}{-1}
\chapter{Cronograma}
\label{cap:cronograma}

\section{Tarefas Realizadas}
	\label{sec:tarefas_realizadas}
	\begin{enumerate}
		\item Leitura da bibliografia sobre:
		\begin{itemize}
			\item Vantagens da virtualização de servidores em termos energéticos:
				\cite{barroso:case_energy_proportional}
				\cite{beloglazov:energy_efficient_allocation_virtual_machines}
				\cite{berl:energy_efficient_cloud_computing}
				\cite{feng:green500_encouraging_sustainable_supercomputing}
				\cite{murugesan:harnessing_green_it}
				\cite{vmware:virtualization_right_sizes_it}
				\cite{rivoire:models_metrics_enable_energy_efficient_optimizations}
			\item Arquitetura interna dos simuladores utilizados: CloudSim e WorkflowSim
				\cite{calheiros:cloudsim}
				\cite{chen:workflowsim}
			\item Escalonamento de tarefas em computação em grade (\emph{grid 
			computing})
				\cite{chaves:scheduling_software_requirements}
				\cite{batista:embedding_software_requirements}
		\end{itemize}
		\item Experimentos iniciais com quatro simuladores de computação em 
		nuvem/grade: CloudSim, GridSim, SimGrid e WorkflowSim
		\item Simulação energética de experimentos-controle no PowerWorkflowSim,
		o simulador que será um dos resultados do TCC
		\item Estudo do código fonte C++ produzido no artigo
			\cite{chaves:scheduling_software_requirements} 
		\item Redação de 19 seções da monografia de um total de 29
	\end{enumerate}

\section{Tarefas em Andamento}
\label{sec:tarefas_andamento}
	\begin{enumerate}
		\item Implementação do algoritmo HEFT
		\item Desenvolvimento do PowerWorkflowSim, que une as funcionalidades
		da biblioteca de simulação energética do CloudSim com o processamento de
		workflows científicos do WorkflowSim
	\end{enumerate}

\section{Tarefas a fazer}
\label{sec:tarefas_a_fazer}
	\begin{enumerate}
		\item Implementar requisitos de software no WorkflowSim
		\item Implementar o algoritmo de \cite{chaves:scheduling_software_requirements} 
		no WorkflowSim
		\item Trabalhar com Elaine Watanabe, aluna do mestrado em Ciência da
		Computação do IME, na concepção de um novo algoritmo 
		de escalonamento
	\end{enumerate}

