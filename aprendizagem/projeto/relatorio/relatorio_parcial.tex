\documentclass[brazil, a4paper]{article}
\usepackage{graphicx}
\usepackage[T1]{fontenc}
\usepackage[utf8]{inputenc}
\usepackage{lmodern}
\usepackage{babel}
\usepackage{url}
\usepackage{listings}
\usepackage{xcolor}
\usepackage{textcomp}

\lstset{language=Python,
	             basicstyle=\footnotesize,
	             numbers=left,
	             numberstyle=\footnotesize,
	             frame=shadowbox,
	             rulesepcolor=\color{blue},
	             showspaces = false, 
		     showstringspaces = false,
	             showtabs = false,
	             }
\begin{document}


\title{Relatório Parcial - Projeto de Aprendizagem Computacional}
\author{Camila Fernandez Achutti \hfill 6795610\\
		Pedro Paulo Vezzá Campos \hfill 7538743}
\date{\today}

\maketitle

%\begin{abstract}
%\end{abstract}


\section{Proposta escolhida}
O grupo optou pela proposta 2, animação de algoritmos de aprendizagem
computacional. O algoritmo que está sendo implementado e simulado visualmente é o
Perceptron.

\section{Sobre o problema de aprendizagem}
\subsection{PERCEPTRON}
Como explicado no enunciado o Perceptron é um algoritmo de classificação 
supervisionada. Sua principal característica e limitação é que apenas possui 
resultados satisfatórios para a classificação de conjuntos linearmente
separáveis.

Assim, instâncias interessantes para o classificador são conjuntos de dados que
podem ou não ser classificados utilizando uma reta para dividir as categorias
de elementos. O caso clássico de conjunto não linearmente separável é a
função ou-exclusivo (XOR). Dentro dos problemas linearmente separáveis,
é interessante considerar casos em que os pontos de dados estão muito ou
pouco separados para verificar a eficiência do classificador.

Neste trabalho esperamos implementar o perceptron para dimensão arbitrária d $\ge$ 2.

\section{Interface Gráfica}

Está sendo desenvolvida uma interface gráfica de visualização da execução do
algoritmo em 2D. Para entradas com dimensões maiores que 2 representaremos
apenas 2 delas, dando ao usuário a opção de escolher qual visualizar.

A implementação está sendo feita em Python com o apoio das bibliotecas
matplotlib (pyplot) para o plot dos gráficos e a interface do usuário usando
PyGTK.

A entrada pode ser efetuada de 3 maneiras diferentes:
\begin{itemize}
\item marcar os pontos no canvas através de click,
\item ler de um arquivo de entrada, 
\item gerar pontos de forma aleatória.
\end{itemize}

Com o dataset de entrada o algoritmo de aprendizagem entra em jogo e se implementado de forma adequada usa os pontos de amostragem como um conjunto de treinamento para tentar descobrir qual é a linha mais adequada para realizar a classificação. 

A animação mostra como o algoritmo evolui e quais são os ajustes feitos para alcançar o resultado final.

A ferramenta de animação terá as seguintes funcionalidades básicas:
\begin{itemize} 
\item Seleção do modo de entrada: Arquivo, Sorteio, Clique;
\begin{itemize}
\item Arquivo: serão necessários a submissão de dois arquivos diferentes, um de dados de treino e outro de testes. Ambos são csv's de 3 colunas. 
\item Sorteio: dados aleátorios são gerados, somente com o controle de serem gerados dois conjuntos linearmente separáveis.

\end{itemize}
\item Botão de 'TREINAR' para que o algoritmo comece a rodar, ao final do algorimo ele pode ser novamente apertado para que o treinamento seja refeito;
\item Botão 'TESTAR' para que o conjunto de teste seja plotado e o resultado possa ser validado pelo usuário;
\item Possibilidade de  ajustar o valor de $\eta$ (taxa de aprendizagem) para controlar a rapidez com que o perceptron aprende;
\item Definição do número máximo de iterações, onde uma iteração representa a atualização do vetor de pesos a cada ponto que é analisado, sendo assim, se a reta inicial arbitrária for correta, encontramos a solução em 0 iterações do algoritmo.
\end{itemize}

A taxa de aprendizagem é a que dimensiona o vetor de treinamento antes de ser adicionado ao vetor de peso durante as atualizações. Experimente valores diferentes, se quiser, valores maiores afetará a taxa de convergência, mas não a própria convergência. 

O número de iterações é colocado de modo que o algoritmo não será executado para sempre se os vetores de entrada não são separáveis. 

Normalmente, um valor maior de $\eta$ fará com que ele encontre uma solução mais rapidamente, mas pode causar a perda de soluções de casos difíceis. E a definição do número máximo de iteração garante que o algoritmo vai ter fim, ainda que não encontre um solução de erro global nulo.

A sequência esperada do usuário e da animação é a seguinte:
\begin{enumerate}
\item seleção do modo de entrada
\item plotagem do conjunto de treinamento 
\item click do usuário no botão 'TREINAR'
\item plotagem das retas que o algoritmo encontrou durante a execução 
\item click no botão 'TESTAR' para que o conjunto de teste seja plotado e o algoritmo possa ser avaliado pelo usuário.
\end{enumerate}
O botão de 'REINICIAR' pode ser usado para que o algoritmo refaça o treinamento com os mesmos dados, mas com a possibilidade de trocar os parâmetros.

\section{Cronograma Proposto}
Na tabela a seguir estão listadas as tarefas a serem cumpridas para o projeto
de MAC0460 de 2013:

\begin{table}[h]
    \begin{tabular}{|p{5cm}|c|c|c|c|}
    \hline
    Atividade                                                       & setembro & outubro & novembro & dezembro \\ \hline
    Implementar o Perceptron                                        & X        & ~       & ~        & ~        \\
    Buscar na Internet e testar entradas interessantes como exemplo & X        & X       & ~        & ~        \\
    Implementar a interface gráfica                                 & X        & X       & X        & ~        \\
    Escrever o relatório parcial                                    & ~        & X       & ~        & ~        \\ 
    Escrever o relatório final                                      & ~        & ~       & X        & X        \\ \hline
    \end{tabular}
    \caption{Cronograma das tarefas a serem realizadas no semestre}
\end{table}

O cronograma proposto foi cumprido e não sofreu qualquer alteração.

\section{O que já foi feito}

\begin{itemize}
\item {\bf Estudo e implementação do algoritmo perceptron em Python}

Realizamos um estudo detalhado do funcionamento do algoritmo  e realizamos a seguinte implementação simplificada em Python:

\lstinputlisting{perceptron.py}

Nesse algoritmo limitamos o número máximo de iterações em 100 e $\eta$ para efeito de teste.

\item {\bf Busca de exemplos interessantes para teste da animação}

\begin{itemize}
\item XOR
	O exemplo aqui é a função XOR. Nele não é possível traçar uma única reta (função linear) tal que divida o plano de maneira que as saídas com valor 0 ficam situadas de um lado da reta e as com valor 1 do outro. Entretanto, este problema pode ser solucionado com a criação de uma camada intermediária na rede e graficamente com uma estrutura em três (ou mais) dimensões.
	\newpage
		\begin{table}[h]
			\centering
			\begin{tabular}{|c|c|c|}
			\hline
			x1 & x2 & u \\ \hline
			0 & 0 & 0 \\
			0 & 1 & 1 \\
			1 & 0 & 1 \\
			1 & 1 & 0 \\ \hline
			\end{tabular}
			\caption{Conjunto de treinamento do exemplo XOR}
		\end{table}
		
	\item Violeta para exportação
	
	Neste exemplo temos que classificar violetas para serem importadas e as que vão ficar no mercado. 
	
	As violetas são classificadas de acordo com uma escala de intensidade de cor que vai de 1 a 8 e outra escala de textura que vai de 1 a 5.
	
	Definimos que a classe 1 é a de exportacão e a classe 2 é a para mercado interno e temos o seguinte conjunto de teste:
	\begin{table}[h]
			\centering
			\begin{tabular}{|c|c|c|}
			\hline
			cor & textura & classe \\ \hline
			4 &5 & 1 \\
			5 & 4 & 1 \\
			6 & 3 & 1 \\
			7 & 1 & 1 \\
			8 & 2 & 1 \\
			1 & 3 & 2 \\
			1 & 5 & 2 \\
			2 & 2 & 2 \\
			3 & 4 & 2 \\
			4 & 2 & 2 \\ \hline
			\end{tabular}
			\caption{Conjunto de treinamento do exemplo das violetas}
		\end{table}

	\begin{table} [h]
			\centering
			\begin{tabular}{|c|c|c|}
			\hline
			cor & textura & classe \\ \hline
			5 & 1 & 1 \\
			5 
