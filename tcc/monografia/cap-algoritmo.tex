%% ------------------------------------------------------------------------- %%
\chapter{Algoritmo Proposto}
\label{cap:algoritmo}

\section{Proposta Inicial}

\begin{codebox}
\Procname{$\proc{Power-HEFT-Lookahead}()$}
	\li $V \gets \{VmMaisRapida\}$ \Comment VMs usadas ao escalonar
	\li $O \gets \text{os tipos de VMs que podem ser instanciadas}$
	\li Ordene o conjunto de tarefas segundo o critério $rank_u$
   
	\li \While há tarefas não escalonadas
		\li \Do $t \gets \text{a tarefa não escalonada de maior } rank_u$
		\zi
	    \li \Comment Vamos tentar escalonar t em uma VM existente
	    \li \For cada $v$ em $V$:
		    \li \Do	$\proc{EscalonarPowerHEFT}(t, v)$
	    \End
	    
	    \zi
	    \li \Comment Vamos tentar escalonar t em uma nova VM
	    \li \For cada $o$ em $O$:
		    \li \Do	$V \gets V \cup \{o\}$
		    \li Atualize os valores de $rank_u$
		    \li $t \gets \text{a tarefa não escalonada de maior } rank_u$
		    \li $\proc{EscalonarPowerHEFT}(t, o)$
	    \End
	    
	    \li Escalone $t$ na VM que minimiza a energia consumida
	    \li Atualize $V$ e $rank_u$ caso necessário
	\End
\End
\end{codebox}

\begin{codebox}
\Procname{$\proc{EscalonarPowerHEFT}(tarefa, VM)$}
	\li $F \gets \text{filhos diretos da } tarefa \text{ no DAG}$
    \li Escalone $tarefa$ em $VM$
	\li Escalone $F$ utilizando o algoritmo \textsc{HEFT}
	\zi
	\li \Comment A modelagem energética utilizada está descrita em
    	\cite{guerout:energy_aware_simulation}
	\li $energia \gets$ \proc{EstimarEnergiaConsumida()}
		\li Volte para o escalonamento do começo do laço
	\li \Return $energia$
\end{codebox}

\section{Proposta Revisada} % (fold)
\label{sec:proposta_revisada}

\begin{codebox}
\Procname{$\proc{TaskClustering}()$}
	\li \For cada Vm com tarefas alocadas
		\li \Do \If $VmAnterior$ e a $VmAtual$ forem do tipo \{VmMaisLenta\}
			\li \Do Aloque uma nova Vm do tipo \{VmMaisRápida\}
			\li Transfira as tarefas da $VmAnterior$ e $VmAtual$ para a nova Vm
		\End
	\End
\end{codebox}

% section proposta_revisada (end)

\begin{codebox}
\Procname{$\proc{HEFT-DynamicAllocationVm}()$}
\li Aloque uma Vm do tipo \{VmMaisRápida\}
\li	Defina os custos computacionais das tarefas e os custos de comunicação entre as tarefas
\zi com valores médios
\li	Calcule $rank_u$ para todas as tarefas varrendo o grafo de ``baixo para cima'',
	iniciando \\pela tarefa final.
\li Ordene as tarefas em uma lista de escalonamento utilizando uma ordem não
\zi crescente de valores de $rank_u$.
\li 	\While há tarefas não escalonadas na lista
\li 		\Do
				Selecione a primeira tarefa, $t_i$ da lista de escalonamento.
\li 			Calcule o tempo mínimo para execução da tarefa $t_i$ com base nas tarefas
\zi das quais $t_i$ dependa				
\li				\For cada VM $m_k$ no conjunto de VM $(m_k \in P)$
\li 				\Do
						Calcule o tempo mais cedo de conclusão da tarefa  $t_i$,
						considerando que ela execute 
\zi         em $m_k$
					\End
\li				Defina o tempo mais cedo de conclusão da tarefa $t_i$ e o tempo
				de início da VM em que esse tempo foi obtido
\li			\If a Vm escolhida não é do tipo \{VmMaisRápida\}
\li				\Then
					Aloque uma nova VM do tipo \{VmMaisLenta\}
\li				\Else
\li					\If a Vm escolhida não é do tipo \{VmMaisRápida\}
\li 					\Then
							Aloque uma nova Vm do tipo \{VmMaisRápida\}
\li							Migre todas as tarefas da Vm antiga
\li 						Defina a tarefa na nova Vm
\li						\Else
\li							Defina a tarefa $t_i$ para ser executada na Vm que
							minimiza o tempo de
\zi							conclusão desta tarefa
						\End
				\End
			\End
\End
\end{codebox}
