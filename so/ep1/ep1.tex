\documentclass[brazil, a4paper]{scrartcl}
\usepackage{graphicx}
\usepackage[T1]{fontenc}
\usepackage[utf8]{inputenc}
\usepackage{lmodern}
\usepackage{babel}
\usepackage{url}
\usepackage[usenames,svgnames,dvipsnames]{xcolor}
\usepackage{listings}                   % para formatar código-fonte (ex. em Java)

\lstdefinestyle{customc}{
  belowcaptionskip=1\baselineskip,
  breaklines=true,
  frame=L,
  xleftmargin=\parindent,
  language=C,
  showstringspaces=false,
  basicstyle=\footnotesize\ttfamily,
  keywordstyle=\bfseries\color{green!40!black},
  commentstyle=\itshape\color{purple!40!black},
  identifierstyle=\color{blue},
  stringstyle=\color{orange},
}

\lstdefinelanguage{diff}{
  morecomment=[f][\color{blue}]{@@},     % group identifier
  morecomment=[f][\color{red}]<,         % deleted lines 
  morecomment=[f][\color{ForestGreen}]>,       % added lines
  morecomment=[f][\color{magenta}]{---}, % Diff header lines (must appear after +,-)
  morecomment=[f][\color{magenta}]{+++},
}

\lstset{ %
language=C,                  % choose the language of the code
basicstyle=\footnotesize,       % the size of the fonts that are used for the code
numbers=left,                   % where to put the line-numbers
numberstyle=\footnotesize,      % the size of the fonts that are used for the line-numbers
stepnumber=1,                   % the step between two line-numbers. If it's 1 each line will be numbered
numbersep=5pt,                  % how far the line-numbers are from the code
showspaces=false,               % show spaces adding particular underscores
showstringspaces=false,         % underline spaces within strings
showtabs=false,                 % show tabs within strings adding particular underscores
frame=single,	                % adds a frame around the code
framerule=0.6pt,
tabsize=2,	                    % sets default tabsize to 2 spaces
captionpos=b,                   % sets the caption-position to bottom
breaklines=true,                % sets automatic line breaking
breakatwhitespace=false,        % sets if automatic breaks should only happen at whitespace
escapeinside={\%*}{*)},         % if you want to add a comment within your code
backgroundcolor=\color[rgb]{1.0,1.0,1.0}, % choose the background color.
rulecolor=\color[rgb]{0.8,0.8,0.8},
extendedchars=true,
xleftmargin=10pt,
xrightmargin=10pt,
framexleftmargin=10pt,
framexrightmargin=10pt
}

\begin{document}


\title{Exercício Programa 1 -- Relatório}
\subtitle{MAC0422 -- Sistemas Operacionais}
\author{André Spanguero Kanayama \hfill 7156873\\
		Pedro Paulo Vezzá Campos \hfill 7538743}
\date{\today}

\maketitle

%\begin{abstract}
%\end{abstract}

\section{Enunciado}
Para este primeiro exercício-programa de MAC0422 -- Sistemas Operacionais, o 
professor requisitou que os alunos alterassem o comportamento do comando
\texttt{F5} no Minix. Após pressionar esta tecla deveria ser exibido
na tela um resumo da tabela de processos:

\begin{enumerate}
	\item \emph{Process ID}
	\item Tempo de CPU
	\item Tempo de sistema
	\item Tempo dos filhos
	\item Endereço do ponteiro da pilha
	\item Endereço dos segmentos \texttt{data}, \texttt{bss}, \texttt{text} 
\end{enumerate}

\section{Observações}
\begin{itemize}
	\item O professor informou verbalmente aos alunos na aula do dia 16/09 que
	não era mais necessário imprimir a tabela de processos segundo a ordem de
	escalonamento.
	\item A versão do Minix escolhida para este EP foi a versão 3.1.7 devido à
	sua melhor compatibilidade com o VirtualBox e maior semelhança com a versão 
	3.1.0, a utilizada pelo livro-texto da disciplina.
\end{itemize}

\section{Descoberta do comportamento do comando \texttt{F5}}
Após uma busca na Internet por ``\emph{Minix function keys}'', o primeiro
resultado indicou o primeiro arquivo importante para o EP: 
\texttt{/usr/src/servers/is/dmp.c}. Nele, havia uma estrutura bastante
intuitiva, um mapeamento de cada tecla para sua respectiva função. Aqui foi
feita a primeira modificação:

\begin{lstlisting}[language=diff]
struct hook_entry {
	int key;
	void (*function)(void);
	char *name;
} hooks[] = {
	{ F1, 	proctab_dmp, "Kernel process table" },
	{ F2,   memmap_dmp, "Process memory maps" },
	{ F3,	image_dmp, "System image" },
	{ F4,	privileges_dmp, "Process privileges" },
< 	{ F5,	monparams_dmp, "Boot monitor parameters" },
---
>     /*????????????????????????????????????????????????????????*/
>     /*????????????????????????????????????????????????????????*/
> 	{ F5,	pt_dmp, "Print process table" },
>     /*????????????????????????????????????????????????????????*/
>     /*????????????????????????????????????????????????????????*/
	{ F6,	irqtab_dmp, "IRQ hooks and policies" },
	{ F7,	kmessages_dmp, "Kernel messages" },
	{ F8,	vm_dmp, "VM status and process maps" },
	{ F10,	kenv_dmp, "Kernel parameters" },
	{ F11,	timing_dmp, "Timing details (if enabled)" },
	{ SF1,	mproc_dmp, "Process manager process table" },
	{ SF2,	sigaction_dmp, "Signals" },
	{ SF3,	fproc_dmp, "Filesystem process table" },
	{ SF4,	dtab_dmp, "Device/Driver mapping" },
	{ SF5,	mapping_dmp, "Print key mappings" },
	{ SF6,	rproc_dmp, "Reincarnation server process table" },
	{ SF8,  data_store_dmp, "Data store contents" },
	{ SF9,  procstack_dmp, "Processes with stack traces" },
};
\end{lstlisting}

Vasculhando o diretório \texttt{/usr/src/servers/is} verificamos que todo o EP
pode ser feito através de modificações no \emph{Information Server} (IS).

\section{Desenvolvimento da função \texttt{pt\_dmp}}
A função \texttt{pt\_dmp}, de \emph{process table dump}, é o trecho de código
principal do EP. Ela foi derivada da função \texttt{mproc\_dmp}, que já estava
implementada no Minix e é responsável por exibir a tabela de processos do
\emph{Process Manager} (PM).

Passamos a estudar as diferentes tabelas de processos existentes no Minix. A
primeira relevante para este trabalho é a tabela de processos do PM, definida
no arquivo \texttt{/usr/src/servers/pm/mproc.h}. Nela, encontramos parte das
informações necessárias ao EP:

\begin{description}
	\item[\emph{Process ID}] Definido no campo \texttt{pid\_t mp\_pid};
	\item[Tempo dos filhos] Definido no campo \texttt{clock\_t mp\_child\_utime;}
\end{description}

Ao perceber que as outras informações necessárias não estavam disponíveis neste
local passamos a vasculhar o código em busca de outras tabelas úteis. Após ver o
código de impressão da tabela de processos do kernel que está no arquivo
\texttt{/usr/src/servers/is/kernel\_dmp.c} encontramos o arquivo \emph{header}
da tabela de processos em \texttt{/usr/src/kernel/proc.h}. Neste arquivo estavam
presentes as outras informações necessárias:

\begin{description}
	\item[Tempo de CPU] Disponível no campo \texttt{clock\_t p\_user\_time}
	\item[Tempo de sistema] Disponível no campo \texttt{clock\_t p\_sys\_time}
	\item[Endereço do ponteiro da pilha] Disponível no campo \texttt{p\_memmap[S].mem\_phys}
	\item[Endereço do segmento \texttt{data}] Disponível no campo \texttt{p\_memmap[D].mem\_phys}
	\item[Endereço do segmento \texttt{text}] Disponível no campo \texttt{p\_memmap[T].mem\_phys}
	\item[Endereço do segmento \texttt{bss}] O endereço do segmento \texttt{bss} 
		é definido como sendo o primeiro endereço
		de memória após o segmento \texttt{data}. Tentamos calcular este endereço
		usando endereços virtuais com a seguinte fórmula:
		
		\texttt{p\_memmap[D].mem\_vir + p\_memmap[D].mem\_len}
		
		Em seguida mapeando-o para o seu
		respectivo endereço físico através da função \texttt{sys\_umap}.
		Porém, a execução chamada de sistema foi negada. O \emph{kernel}
		informava o seguinte erro: 
		
		\texttt{SYSTEM: denied request 14 from 73137.}
		
		Assim, contornamos este problema realizando o cálculo através da
		seguinte fórmula:
		
		\texttt{p\_memmap[D].mem\_phys + p\_memmap[D].mem\_len}
		
		
\end{description}

É importante ressaltar que apesar de representarem os mesmos processos
(salvo algumas exceções), as
entradas de ambas tabelas não estão na mesma ordem. Para resolver este problema
o próprio Minix apresenta um mapeamento na tabela de processos do \emph{kernel}
de uma entrada nesta tabela para a tabela de processos do PM. Esta informação 
foi descoberta após vasculhar o código do programa \texttt{top}. O campo
relevante é o campo \texttt{p\_nr}. Processos gerenciados pelo \emph{kernel}
mas não pelo PM possuem o campo \texttt{p\_nr} $< 0$. Dessa forma, definimos a 
regra que para um dado processo \texttt{proc[i]} com \texttt{proc[i].p\_nr} $\ge
0$ e $0 \le$ \texttt{i} $<$ \texttt{NR\_PROCS} é equivalente ao processo
\texttt{mproc[proc[i].p\_nr]}. Para garantir que esse mapeamento é
válido, fizemos um teste imprimindo o nome dos processos equivalentes em ambas
tabelas. O experimento foi bem sucedido.


O código final para a função \texttt{pt\_dmp} foi inserido no arquivo
\texttt{/usr/src/servers/is/dmp\_pm.c} e está descrito abaixo:

\begin{lstlisting}[style=customc]
/*?????????????????????????????????????????????????????????????????*/
/*?????????????????????????????????????????????????????????????????*/

PUBLIC void pt_dmp()
{
  struct mproc *mp;
  int i, n=0;
  int result;
  phys_bytes p;
  static int prev_i = 0;

  struct proc *pr;

  printf("Process table\n");

  getsysinfo(PM_PROC_NR, SI_PROC_TAB, mproc);
  sys_getproctab(lproc);

  printf("-pid- -cpu_t- -sys_t- -chld_t- ---stackpointer ---data--- ---bss---  ---text---\n");
  for (i=prev_i; i<NR_PROCS; i++) {
    pr = &lproc[i];
    if(pr->p_nr < 0)
        continue;
    mp = &mproc[pr->p_nr];
  	if (mp->mp_pid == 0 && i != PM_PROC_NR) continue;
  	if (++n > 22) break;

  	printf("%4d  %6d %7d %8d      0x%08X  0x%08X 0x%08X 0x%08X", 
  		mp->mp_pid,
  		pr->p_user_time,
  		pr->p_sys_time,
  		mp->mp_child_utime,
  		pr->p_memmap[S].mem_phys,
  		pr->p_memmap[D].mem_phys,
  		pr->p_memmap[D].mem_phys + pr->p_memmap[D].mem_len,
  		pr->p_memmap[T].mem_phys);
  	printf("\n");
  }
  if (i >= NR_PROCS) i = 0;
  else printf("--more--\r");
  prev_i = i;
}
/*?????????????????????????????????????????????????????????????????*/
/*?????????????????????????????????????????????????????????????????*/
\end{lstlisting}

Por fim, foi necessário editar o arquivo de protótipos de funções do IS
para incluir a nova função criada. A edição foi feita no arquivo 
\texttt{/usr/src/servers/is/proto.h} e está reproduzida abaixo:
\begin{lstlisting}[language=diff]
/* dmp_pm.c */
_PROTOTYPE( void mproc_dmp, (void)					);
_PROTOTYPE( void sigaction_dmp, (void)					);
> /*?????????????????????????????????????????????????????????????*/
> /*?????????????????????????????????????????????????????????????*/
> _PROTOTYPE( void pt_dmp, (void)					);
> /*?????????????????????????????????????????????????????????????*/
> /*?????????????????????????????????????????????????????????????*/
\end{lstlisting}

\end{document}

