%% ------------------------------------------------------------------------- %%
\chapter{Conclusões}
\label{cap:conclusoes}

Nesta monografia apresentada como parte da disciplina Trabalho de Formatura
Supervisionado foram estudadas e experimentadas diversas técnicas de
escalonamento de tarefas em computação em nuvem. O trabalho contou com a
orientação do professor Daniel Macêdo Batista.

No Capítulo \ref{cap:introducao} foi vista uma motivação para o estudo de
técnicas de economia de energia para computação em nuvem. Notou-se um
avanço no interesse por uma computação mais econômica no uso de recursos. Ao
mesmo tempo, foram definidos os objetivos desta monografia e os seus desafios
relacionados, tais como o fato que otimizar o tempo de processamento de uma
aplicação paralela é NP-difícil.

O Capítulo \ref{cap:conceitos} pontuou conceitos necessários para a compreensão
desta monografia. Foram estudados assuntos como consumo energético e técnicas
atuais de economia de energia, um algoritmo clássico da área, o
\emph{Heterogeneous Earliest Finish Time} e, por fim, simuladores de computação
em nuvem, necessários para os experimentos desenvolvidos posteriormente.

Continuando, no Capítulo \ref{cap:algoritmo} foram apresentadas duas propostas
de algoritmos de escalonamento com um foco na eficiência energética, o PowerHEFT
e o \emph{HEFT Dynamic Allocation of VM}. Em conjunto, foram traçadas as
motivações e intuições utilizadas na concepção destes algoritmos.

Por fim, no Capítulo \ref{cap:experimentos} foram apresentados os
desenvolvimentos técnicos da monografia, com a tentativa de desenvolvimento de
um simulador adaptado para a monografia, as configurações e dados de entrada
escolhidos para os experimentos e, por fim, os resultados obtidos e uma
discussão.

Esta monografia trouxe algumas contribuições relevantes. Na parte acadêmica,
foram desenvolvidos dois novos algoritmos de escalonamento com foco energético.
Resultados experimentais mostram que para entradas menores ambos são
tão eficientes quanto os outros algoritmos estudados, o que não se concretizou
para o caso de entradas maiores.

Já na parte técnica, o autor do trabalho implementou o \emph{HEFT Dynamic
Allocation of VM} em dois simuladores de computação em nuvem e disponibilizou o 
código fonte para os desenvolvedores originais. Por fim, o autor colaborou
com diversos pesquisadores internacionais, o que resultou na disponibilização
do código fonte de um simulador e na adição do autor como contribuidor de outro
simulador.

\section{Trabalhos Futuros} % (fold)
\label{sec:trabalhos_futuros}

O algoritmo PowerHEFT mostrou-se pouco escalável no critério de \emph{makespan}.
Isto pode ser melhorado de diversas formas, como por exemplo, a definição de um
\emph{deadline} para o tempo de execução da aplicação paralela.

A área de escalonamento energeticamente eficiente é muito recente, como é
possível ver pelas datas de publicação dos artigos estudados nesta monografia.
Uma revisão bibliográfica mais extensa poderia ser feita para encontrar novas
propostas de algoritmos que ainda apresentassem margem para melhorias e, 
em seguida, trabalhar em novas propostas de algoritmos.

Por fim, os simuladores utilizados ainda são muito novos e pouco testados.
Foram identificados alguns problemas de engenharia de software, tais como falta
de documentação, código morto, problemas de tradução, etc. que poderiam ser
corrigidos.

% section trabalhos_futuros (end)