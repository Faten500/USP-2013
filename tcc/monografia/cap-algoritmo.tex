%% ------------------------------------------------------------------------- %%
\chapter{Algoritmo Proposto}
\label{cap:algoritmo}

Após a revisão bibliográfica, foi decidido que os algoritmos de referência para
comparação com a proposta desenvolvida neste trabalho foram o algoritmo HEFT,
assunto da seção \ref{subsec:heft} e os algoritmos apresentados no trabalho de
Tom Guérout et al \cite{guerout:energy_aware_simulation}. O desenvolvimento de
uma proposta de algoritmo de escalonamento de fluxos de trabalho que fosse
energeticamente eficiente foi fruto da observação de alguns conceitos
importantes:

Uma ideia inicial seria alterar a fase de seleção do algoritmo HEFT para que, ao
invés de tentar minimizar o tempo mais cedo de conclusão de cada tarefa
processada, tentasse minimizar o consumo energético. Porém, devido à natureza de
algoritmo guloso do HEFT, esta ideia foi abandonada.

Para compreender esta decisão é preciso lembrar de um fato. A eficiência
energética máxima de um processador (Que pode ser medida em energia consumida
por instrução processada) acontece quando este processador opera no máximo de
sua capacidade \cite{barroso:case_energy_proportional}. Desta forma, quando o
algoritmo HEFT tentar otimizar o consumo energético de cada tarefa
individualmente ele optaria por sempre manter a máquina escolhida operando na
frequência máxima. De fato, isto é eficiente para um processamento sequencial,
mas não necessariamente se mantém verdade para aplicações paralelas. Como
maneira de remediar esta possível vulnerabilidade do algoritmo, foi desenvolvida
uma versão do algoritmo que emprega uma estratégia de \emph{lookahead}, descrita
na seção \ref{sec:proposta_inicial}.

Paralelamente, foi desenvolvida uma segunda proposta, que faz uso da
característica do CloudSim\_DVFS que permite que VMs sejam criadas ou removidas
durante o momento do escalonamento. Uma variante desenvolvida emprega, ainda,
uma estratégia de \emph{clustering} inspirada no trabalho desenvolvido para o
WorkflowSim. A descrição detalhada da segunda proposta está na seção
\ref{sec:heft_davm}.

\section{PowerHEFT}
\label{sec:proposta_inicial}

Como explicado na introdução do capítulo, a primeira proposta de algoritmo
energeticamente eficiente, chamada de PowerHEFT, é uma adaptação do algoritmo
HEFT para empregar uma estratégia de \emph{lookahead}. Sua inspiração veio da
leitura do artigo \cite{bittencourt:heft_lookahead} que analisa diversas
otimizações ao HEFT. Uma das propostas é uma mudança na fase de seleção do
algoritmo: ao invés de escolher a máquina que minimize o tempo mais cedo de
conclusão de cada tarefa sozinha, seria escolhida a máquina que minimizasse o
tempo mais cedo de conclusão das tarefas filhas, após um escalonamento
``simulado'' com o HEFT. Esta proposta foi chamada de \textsc{HEFT-Lookahead}.

O PowerHEFT funciona da seguinte maneira: como não é sabido de antemão quantas
ou quais VMs são necessárias para um processamento que otimize o consumo
energético, o algoritmo é foi projetado para alocar novas VMs sob demanda.
Inicialmente, há apenas uma máquina alocada, a mais rápida disponível. Esta
máquina é responsável por receber as primeiras tarefas a serem escalonadas,
pertencentes a um possível caminho crítico, como comentado na seção
\ref{subsec:heft}.

Para cada tarefa a ser processada, o algoritmo ``força'' a alocação de cada
tarefa em cada máquina disponível. Em seguida, o algoritmo HEFT é invocado para
escalonar todas as tarefas filhas da tarefa atual. Neste momento, o
escalonamento parcial possui o seu consumo energético avaliado, segundo a
modelagem energética do trabalho \cite{guerout:energy_aware_simulation} e
comentada na Seção \ref{ssub:modelagem_energetica}. A melhor opção de máquina no
momento é memorizada.

Em um segundo momento do processamento da tarefa, o algoritmo analisa se seria
mais vantajoso energeticamente alocar mais uma máquina para auxiliar o
processamento. Neste momento, uma nova máquina é disponibilizada, sendo
comparadas as diferentes opções disponíveis para alocação. Em seguida, o
algoritmo repete o procedimento descrito no parágrafo anterior.

Uma vez que é conhecida a melhor opção de máquina, o algoritmo aplica o
escalonamento para a tarefa atual e prossegue para a próxima tarefa da lista de
escalonamento.

É importante ressaltar que algumas tarefas filhas da tarefa atual podem não
estar prontas para serem escalonadas por estarem aguardando alguma outra
tarefa pai ainda não processada. Nestes casos, a fase \emph{lookahead} ignora
a existência destas tarefas pai. Isto tem como consequência o fato que a
estimativa energética pode ser mais otimista que a realidade.

O algoritmo do PowerHEFT está descrito abaixo:

\begin{codebox}
\Procname{$\proc{EscalonarPowerHEFT}(tarefa, VM)$}
	\li $F \gets \text{filhos diretos da } tarefa \text{ no DAG}$
    \li Escalone $tarefa$ em $VM$
	\li Escalone $F$ utilizando o algoritmo \textsc{HEFT}
	\zi
	\li \Comment A modelagem energética utilizada está descrita em
    	\cite{guerout:energy_aware_simulation}
	\li $energia \gets$ \proc{EstimarEnergiaConsumida()}
		\li Volte para o escalonamento do começo do laço
	\li \Return $energia$
\end{codebox}


\begin{codebox}
\Procname{$\proc{Power-HEFT-Lookahead}()$}
	\li $V \gets \{VmMaisRapida\}$ \Comment VMs usadas ao escalonar
	\li $O \gets \text{os tipos de VMs que podem ser instanciadas}$
	\li Ordene o conjunto de tarefas segundo o critério $rank_u$
   
	\li \While há tarefas não escalonadas
		\li \Do $t \gets \text{a tarefa não escalonada de maior } rank_u$
		\zi
	    \li \Comment Vamos tentar escalonar t em uma VM existente
	    \li \For cada $v$ em $V$:
		    \li \Do	$\proc{EscalonarPowerHEFT}(t, v)$
	    \End
	    
	    \zi
	    \li \Comment Vamos tentar escalonar t em uma nova VM
	    \li \For cada $o$ em $O$:
		    \li \Do	$V \gets V \cup \{o\}$
		    \li Atualize os valores de $rank_u$
		    \li $t \gets \text{a tarefa não escalonada de maior } rank_u$
		    \li $\proc{EscalonarPowerHEFT}(t, o)$
	    \End
	    
	    \li Escalone $t$ na VM que minimiza a energia consumida
	    \li Atualize $V$ e $rank_u$ caso necessário
	\End
\End
\end{codebox}


\section{\emph{HEFT Dynamic Allocation of VM}} % (fold)
\label{sec:heft_davm}

A segunda proposta de algoritmo baseia-se nas características do simulador
CloudSim\_DVFS descrito na Seção \ref{subsec:cloudsim_dvfs}. Aqui, a motivação para
o algoritmo é explorar ao máximo o paralelismo na execução do fluxo de trabalho.
Novamente, o conjunto de máquinas utilizadas é inicializado com apenas uma VM.

Para cada tarefa a ser escalonada é calculado seu tempo mais cedo de
\emph{início}. Este instante marca o primeiro momento que a tarefa pode ser
executada. Em seguida, o algoritmo HEFT é aplicado para determinar qual será o
tempo de início efetivo da tarefa, que depende da ocupação atual das máquinas
disponíveis. Sempre que for detectada uma sobrecarga nas máquinas, indicada por
$\text{tempo de início } \neq \text{ tempo mais cedo de início}$, uma nova VM (A
mais lenta/econômica disponível) é alocada. A esperança é que isto garanta uma
economia na alocação de VMs sem prejudicar o \emph{makespan} da aplicação.

Uma segunda otimização é a aplicação de \emph{clustering} inspirada no trabalho
\cite{chen:workflowsim}. Sempre que uma máquina for utilizada para mais de uma
tarefa no escalonamento esta máquina possui sua configuração alterada da mais
lenta para a mais rápida disponível. O algoritmo do \emph{HEFT Dynamic Allocation
of VM} está descrito abaixo:

\begin{codebox}
\Procname{$\proc{TaskClustering}()$}
	\li \For cada Vm com tarefas alocadas
		\li \Do \If $VmAnterior$ e a $VmAtual$ forem do tipo \{VmMaisLenta\}
			\li \Do Aloque uma nova Vm do tipo \{VmMaisRápida\}
			\li Transfira as tarefas da $VmAnterior$ e $VmAtual$ para a nova Vm
		\End
	\End
\end{codebox}

\begin{codebox}
\Procname{$\proc{HEFT-DynamicAllocationVm}()$}
\li Aloque uma Vm do tipo \{VmMaisRápida\}
\li	Defina os custos computacionais das tarefas e os custos de comunicação entre as tarefas
\zi com valores médios
\li	Calcule $rank_u$ para todas as tarefas varrendo o grafo de ``baixo para cima'',
	iniciando \\pela tarefa final.
\li Ordene as tarefas em uma lista de escalonamento utilizando uma ordem não
\zi crescente de valores de $rank_u$.
\li 	\While há tarefas não escalonadas na lista
\li 		\Do
				Selecione a primeira tarefa, $t_i$ da lista de escalonamento.
\li 			Calcule o tempo mínimo para execução da tarefa $t_i$ com base nas tarefas
\zi das quais $t_i$ dependa				
\li				\For cada VM $m_k$ no conjunto de VM $(m_k \in P)$
\li 				\Do
						Calcule o tempo mais cedo de conclusão da tarefa  $t_i$,
						considerando que ela execute 
\zi         em $m_k$
					\End
\li				Defina o tempo mais cedo de conclusão da tarefa $t_i$ e o tempo
				de início da VM em que esse tempo foi obtido
\li			\If a Vm escolhida não é do tipo \{VmMaisRápida\}
\li				\Then
					Aloque uma nova VM do tipo \{VmMaisLenta\}
\li				\Else
\li					\If a Vm escolhida não é do tipo \{VmMaisRápida\}
\li 					\Then
							Aloque uma nova Vm do tipo \{VmMaisRápida\}
\li							Migre todas as tarefas da Vm antiga
\li 						Defina a tarefa na nova Vm
\li						\Else
\li							Defina a tarefa $t_i$ para ser executada na Vm que
							minimiza o tempo de
\zi							conclusão desta tarefa
						\End
				\End
			\End
\End
\end{codebox}


% section proposta_revisada (end)