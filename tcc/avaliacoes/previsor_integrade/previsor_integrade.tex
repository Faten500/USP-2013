\documentclass{article}
\usepackage[brazilian]{babel}
\usepackage[T1]{fontenc}
\usepackage[utf8]{inputenc}
\usepackage{lmodern}
\usepackage[a4paper]{geometry}
%\usepackage{url}
%\usepackage{graphicx}
%\usepackage[pdfborder={0 0 0}]{hyperref}
%\usepackage{amssymb}
%\usepackage{amsmath}

\begin{document}


\author{Pedro Paulo Vezzá Campos}
\title{Avaliação do TCC \\ ``Previsão de Utilização de Recursos por Aplicações no InteGrade''}
\date{\today}
\maketitle

\section{Dados do Trabalho Analisado}
	\begin{description}
		\item[Título] Previsão de Utilização de Recursos por Aplicações no InteGrade
		\item[Ano] 2009
		\item[Aluno] Fabio Augusto Firmo
		\item[Orientador] Marcelo Finger
		\item[Nota Obtida] Não consta
	\end{description}

\section{Resumo da Monografia}
	O trabalho propõe uma melhoria em um software de gerenciamento de grades computacionais oportunistas (Aproveitando capacidade computacional ociosa), o \emph{InteGrade}. O problema a ser solucionado é que o programa não é capaz de realizar uma previsão do tempo de processamento e consumo de memória de uma aplicação a ser processada na grade. No momento o criador da aplicação é que fornece tais valores, baseados em suposições ou conhecimento prévio de dados. Há a opção, também, de não fornecer previsão nenhuma. A ausência ou estimativa ruim de valores compromete o desempenho da grade já nestes podem acontecer dois problemas: Estimativas muito conservadorras subutilizam a capacidade da grade enquanto que se estas forem muito agressivas pode haver interferência em outros usos dos computadores que compõem o \emph{cluster}, prejudicando o objetivo de ser uma grade oportunista.

	A solução proposta é baseada em dois aspectos: Primeiro, a definição e busca por execuções anteriores de tarefas similares como parâmetro para a previsão para a tarefa atual. Segundo, a definição de uma fórmula que possa mapear um histórico de execuções na previsão propriamente dita. O trabalho estudou opções estatisticamente simples para modelar os problemas, visando uma implementação igualmente simples. Como critério de similaridade foram escolhidas execuções da mesma aplicação e como critério para consumo computacional a mediana foi a melhor classificada. Comparando os resultados com a bibliografia, os resultados encontram-se em um patamar intermediário. O código relativo ao TCC foi adicionado em um \emph{branch} do desenvolvimento do \emph{InteGrade}.
	
\section{Avaliação da parte técnica}
	A redação do trabalho é simples e clara. Não há grandes digressões o que contribui para que ele seja bem enxuto, com 31 páginas no total. Encontrei poucos erros ortográficos, menos de 5, frutos de erros de digitação. A organização do trabalho ficou um pouco confusa, as seções 3.1, 3.1.1 -- 3.1.3 poderiam vir antes do capítulo 2 como forma de fornecer as informações necessárias ao restante do texto. No entanto, os conceitos são suficientemente claros para garantir que a leitura ocorra sem problemas.

	Em vários momentos o autor, ao se deparar com uma decisão de projeto, acabou por optar pela solução mais simples. Vale ressaltar que o tempo que o autor deve ter utilizado para se adaptar com a tecnologia que estava trabalhando, \emph{CORBA} e a arquitetura do \emph{InteGrade}, não deve ter sido trivial, o que deve ser levado em conta na análise dos resultados obtidos. De fato, isso pode ter garantido que o trabalho fosse concluído a tempo, por outro lado, não foi possível perceber em sua monografia algum ponto no qual ele tenha sido mais audacioso. 
	
	A seção de testes do software é a única mais fraca do trabalho. Foram desenvolvidos alguns testes unitários mas não foram produzidos testes de aceitação ou regressão, de acordo com o autor, pela falta de tempo. Isso é uma deficiência do trabalho, software mal testado é uma porta para diversos bugs obcuros.
	
\section{Avaliação da parte subjetiva}
	Foi feito um balanço breve do curso e foram apresentadas as disciplinas mais úteis na opinião do autor. O balanço poderia ser mais detalhado, o aluno resumiu quatro anos de bacharelado em três parágrafos apenas. Ainda, foi apresentado um pouco do trabalho de bastidores na confecção do TCC, algo muito útil para dar uma perspectiva ao leitor das dificuldades enfrentadas no desenvolvimento da monografia, esta ideia parece bastante útil de ser incorporada à minha monografia. 
	
\section{Comentários}
	No geral simpatizo com o trabalho por ser objetivo e sincero. Os resultados são intermediários aos da bibilografia consultada, o que é um ótimo resultado para um TCC. Particularmente concederia uma nota $9.0$, com os únicos descontos sendo a deficiência nos testes do software e a falta de alguma parte mais audaciosa no trabalho.


%\nocite{*}
%\bibliographystyle{abnt-num}
%\bibliography{bibliografia}

\end{document}
