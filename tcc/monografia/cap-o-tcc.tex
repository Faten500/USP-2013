%% ------------------------------------------------------------------------- %%
\chapter{O Trabalho de Conclusão de Curso}
\label{cap:o_tcc}

Este capítulo é dedicado a comentar alguns dos aspectos vivenciados durante a
produção deste trabalho de formatura. Inicialmente, na Seção
\ref{sec:desafios_frustracoes} há uma discussão sobre os desafios e frustrações
enfrentados e em seguida, na Seção \ref{sec:observacores_sobre_aplicacao_real}
são feitas observações sobre a aplicação futura dos resultados alcançados.

\section{Desafios e frustrações}
\label{sec:desafios_frustacoes}

Um trabalho a ser desenvolvido durante um ano inteiro e primariamente de maneira
individual naturalmente enfrenta alguns problemas. É tarefa do autor enfrentá-los
e tentar fazer o melhor proveito possível deles. As principais dificuldades
estão relacionadas abaixo.

\subsection{Escopo} % (fold)
\label{sub:escopo}

Controlar o escopo do trabalho é algo fundamental para garantir que o trabalho
seja concluído a tempo e sem grandes sustos. Naturalmente, ao comparar os
resultados obtidos com a proposta inicial do TCC pode-se perceber como a
proposta é megalomaníaca. Caso a ideia de adaptar o CloudSim para aceitar DAGs
como entrada fosse levada adiante, muito provavelmente o TCC não seria sobre um
novo algoritmo de escalonamento em computação em nuvem mas sim sobre o trabalho
de engenharia de software desenvolvido no ano.

Para gerenciar este problema, pesquisas constantes por material já desenvolvido
por outras pessoas foi fundamental para que fosse possível manter o foco
no que realmente é o cerne do trabalho. Mas é importante ressaltar que há sempre
um compromisso: uma economia de tempo conseguida ao descobrir uma ferramenta que
auxilia o seu trabalho é compensado com um aumento de tempo gasto para estudá-la
e adaptá-la. Nem sempre o saldo é positivo, como por exemplo com o tempo gasto
tentando adaptar o WorkflowSim sem sucesso antes de migrar para o CloudSim\_DVFS.

% subsection escopo (end)

\subsection{Tempo} % (fold)
\label{sub:tempo}

É incrível como coisas aparentemente rápidas de serem concluídas podem tomar
mais tempo que o planejado. Fazer boas estimativas de tempo para concluir uma
tarefa está longe de ser uma ciência exata. Alguns aprendizados conseguidos com
o TCC:

\begin{description}
	\item[Chaveamento de contexto] Tentar fazer muitas coisas ao mesmo tempo é
		uma receita para a ineficiência. O \emph{overhead} presente ao mudar
		a mente de uma tarefa para outra é uma importante fonte de tempo gasto
		sem nenhum resultado prático.
	\item[Controle do tempo] Durante a monografia foi aplicada parcialmente a
		Técnica Pomodoro \footnote{Para mais informações,
		visite \url{http://pomodorotechnique.com/}} os grandes resultados foram
		controlar o tempo de procrastinação voluntário e involuntário e ainda
		dividir o tempo de trabalho em parcelas que evitem a fadiga mental
		adquirida ao manter a concentração por muito tempo em um assunto.
\end{description}

% subsection tempo (end)

\section{Observações sobre a aplicação real de conceitos estudados}
\label{sec:observacoes_sobre_aplicacao_real}

Este TCC possui um enfoque bastante teórico na área de escalonamento. Os
experimentos foram realizados em simuladores, o que garantiu uma economia
financeira e maior controle nos resultados. Porém, uma vez que a parte
conceitual esteja sólida e os resultados sejam promissores, é possível
transportar os trabalhos para ambientes mais realistas. Em última análise,
provedores de computação em nuvem  são empresas, interessadas por tornar seus
processos mais eficientes. E assim, este trabalho possui grande potencial de
aplicação prática.

Seja seguindo em uma carreira acadêmica ou na indústria, o aprendizado obtido
com os simuladores e os algoritmos trabalhados durante o ano de 2013 não será
em vão. Esse pode ser aproveitado em novos experimentos seja na área de
escalonamento e energia, seja com outro enfoque.

\section{Colaboração na produção do trabalho} % (fold)
\label{sec:colaboracao_na_producao_do_trabalho}

O desenvolvimento deste TCC teve como ponto forte, a colaboração com
pesquisadores internacionais, tai como Weiwei Chen, Rodrigo Neves Calheiros e
Guérout Tom. O processo de vencer a timidez para entrar em contato com com estas
pessoas foi crucial para o avanço dos trabalhos.

Em todos os contatos os pesquisadores foram bastante solícitos, dispostos a
colaborar com os trabalhos. Alguns dos resultados foram a inclusão minha
inclusão como contribuidor do WorkflowSim, implementando o algoritmo HEFT no
simulador e a liberação do código fonte do CloudSim\_DVFS, que por um lapso dos
autores ainda não estava disponível para download.

Questões sociais e profissionais são assuntos ainda muito pouco trabalhando na
graduação, apesar de serem pontos ponto vitais para o sucesso profissional de um
graduado em Computação. Esta monografia ajudou a suprir um pouco esta lacuna.


% section colabora_o_na_produ_o_do_trabalho (end)