 
\documentclass[margin=<value>,11pt]{res} % default is 10 pt
% margin option puts section titles to left of text
%\usepackage{helvetica} % uses helvetica postscript font (download helvetica.sty)
%\usepackage{newcent}   % uses new century schoolbook postscript font 
\usepackage[brazilian]{babel}
\usepackage[a4paper]{geometry}
\usepackage[T1]{fontenc}
\usepackage[utf8]{inputenc}
\usepackage{lmodern}

\begin{document}
\name{PEDRO PAULO VEZZÁ CAMPOS\\[12pt]} % the \\[12pt] adds a blank line after name

\address{R. Indiana, 998 -- São Paulo, SP 04562-001\\
        (11) 97132-1145, (11) 2389-7150 \\ \texttt{pedro@vezza.com.br}}
%\address{ {\bf Permanent Address} \\ 29 Amorous Avenue \\ Madison, NJ 07940 \\
%         (201) 777-3821 }

\begin{resume}
 
\section{OBJETIVO}  
     Trabalhar na área de desenvolvimento de soluções em Cloud Computing e Redes de Computadores.
 
\section{RESUMO}
Sou formando em Ciência da Computação pelo IME-USP, com previsão de formatura para o final de 2013. Possuo grande versatilidade na área de Computação por ter sido exposto a diferentes experiências durante a graduação. Ingressei na UFSC como o primeiro colocado no vestibular 2009 para o curso de Ciências da Computação e 89º geral. A graduação possui renome na área de Engenharia de Software e está consolidando-se em Microeletrônica. Participei de um projeto de iniciação científica nesta última área. Em seguida, fui o único admitido em concurso de transferência para o IME-USP, famoso pelo enfoque matemático e algoritmico. Lá, exercitei estas habilidades como participante da Maratona de Programação ACM-ICPC. Ainda, iniciei a segunda iniciação científica, na área de ensino de Computação, o que me concedeu uma visão ampla da carreira. Pretendo trabalhar na área de desenvolvimento de soluções em Cloud Computing e Redes de Computadores. Sou certificado pela IBM para o uso de tecnologias de Computação em Nuvem. Possuo disponibilidade para mudança, já vivi em sete cidades de quatro estados diferentes.

\section{ESTUDO} 
	{\bf Universidade de São Paulo}\\
	Bacharel em Ciência da Computação \\
	2011 - 2013 \\
	Média Ponderada: 8,7/10 \\
	Graduação prevista para fim de 2013

	{\bf Universidade Federal de Santa Catarina}\\
	Bacharel em Ciência da Computação \\
	2009 - 2010 \\
	Média Ponderada: 9,48/10 \\
	Transferido para a USP \\

\vfill
\section{EXPERIÊNCIA}
	{\bf Universidade de São Paulo} \\
		\begin{ncolumn}{2} % produces two equally spaced columns
		\underline{Bolsista de Iniciação Científica}     &      03/2012 - Atual
		\end{ncolumn}
		Trabalho no projeto Apoio BCC, que visa diminuir a evasão do curso de Ciência da Computação do IME-USP através de trabalhos para contextualizar o curso, analisar currículos de diferentes cursos e realizar pesquisas com alunos egressos do curso. É co-responsável pela aplicação e análise de questionários a ex-alunos com o intuito de buscar tendências no mercado de trabalho, sugestões e críticas ao curso. Estuda currículos nacionais, internacionais e de referência para contribuir no processo de reformulação do currículo de Computação do IME-USP.

	{\bf Universidade Federal do Rio Grande do Sul} \\
		\begin{ncolumn}{2} % produces two equally spaced columns
		\underline{Bolsista de Iniciação Tecnológica (ITI)}     &      03/2010 - 12/2010
		\end{ncolumn}
		Bolsista do projeto SisRas - Sistemas Computacionais com Capacidade de Confiabilidade, Disponibilidade e Utilidade (RAS) gerido pela Universidade Federal do Rio Grande do Sul. A área principal de pesquisa foi o estudo e projeto de somadores rápidos tolerantes a falhas transientes. Outros estudos incluiram a melhoria de ferramentas para simulação de scratchpad memories (SPM). A pesquisa foi realizada no Laboratório de Automação de Projeto de Sistemas na Universidade Federal de Santa Catarina sob a orientação do Prof. Dr. José Luis Güntzel.
 
 
\section{RECONHECIMENTOS}
	{\bf Certificado de Desempenho Acadêmico 2009/1, 2009/2, 2010/1 e 2010/2}\\
	Universidade Federal de Santa Catarina\\
	fevereiro de 2013

\section{IDIOMAS}
	\begin{ncolumn}{3}
	{\bf Português} & {\bf Inglês}        & {\bf Francês} \\
	Nativo    & Intermediário & Básico  \\
	\end{ncolumn}



\section{CERTIFICADOS}
{\bf Certified Solution Advisor - Cloud Computing Architecture V1}\\
IBM\\
setembro de 2011

\end{resume} 
\end{document}





