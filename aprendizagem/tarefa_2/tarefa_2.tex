\documentclass[brazil]{article}
\usepackage{graphicx}
\usepackage[T1]{fontenc}
\usepackage[utf8]{inputenc}
\usepackage{lmodern}
\usepackage{babel}
\usepackage{url}


\begin{document}


\title{Tarefa 2: Revisão da Tarefa 1}
\author{Pedro Paulo Vezzá Campos - 7538743}
\date{\today}

\maketitle

%\begin{abstract}
%\end{abstract}


\section{Definição do problema}
Alunos aspirantes a designer aprendem nos primeiros semestres da graduação
o conceito de \emph{design com tipos}. São apresentadas as principais famílias
tipográficas, tais como serifadas, sem serifa, monoespaçadas etc. Ainda, são 
estudadas algumas fontes famosas, tais como Garamond, Helvetica, Bodoni, etc.

Uma dificuldade recorrente de alunos é de identificar uma fonte em um dado texto. 
Tal problema se resume fundamentalmente a um problema de classificação como visto
em sala de aula. O usuário forneceria uma série de características da fonte que
observa e o sistema o apresentaria as fontes que melhor se encaixam nessas
características.

Note que nem sempre o usuário pode consultar o conjunto completo de caracteres da 
fonte de interesse para fornecer suas características. Um exemplo clássico desse
problema é quando o usuário vê a fonte buscada em um logotipo.


\section{Caracterização dos dados (como representar?)}
Cada fonte do banco de dados é caracterizada como sendo uma tupla de 
\emph{features}. Algumas \emph{features} e seus valores possíveis são 
apresentados abaixo:

\begin{description}
	\item[A fonte possui serifa?] \{Sim, Não\}
	\item[Como é o formato da cauda da letra Q?] \{Cruza o círculo, 
	Toca o círculo, Fica abaixo do círculo e está separada, Está na parte interna
	do círculo, Faz parte de uma forma aberta\}
	\item[Qual a posição da base da letra J em relação à linha de base?] \{Na 
	linha de base, Abaixo da linha de base\}
	\item[Qual o estilo da linha do caractere \$?] \{Uma linha cruzando o S, duas
	linhas cruzando o S, Uma linha que não cruza o S, Duas linhas que não cruzam
	o S\}
	\item[Como é o cruzamento da linha horizontal com a vertical do P?] \{Há um 
	espaço, uma encosta na outra, uma cruza a outra com excesso\}
\end{description}


\section{O que/como seria uma solução (computacional) boa do seu problema?}
O sistema não é de missão crítica e não possui a obrigação de conhecer todas as
fontes criadas até o momento. Apenas a resposta mais provável das fontes 
disponíveis no banco de dados já pode ser uma boa indicação ao usuário de qual
é a fonte desconhecida.

\section{Forma de treinamento}
Para este problema de classificação, cabe uma abordagem \textbf{supervisionada},
de forma a garantir a homogeneidade e qualidade das instâncias de treinamento
para o classificador.


\section{Quais informações de contexto podem ser úteis para o reconhecimento?}
Basicamente cada letra, número e símbolo pode ter características determinadas
para possivelmente facilitar a busca pela fonte correta.

Ainda, algumas \emph{features} não são relevantes dependendo de certas 
``macro-características''. Por exemplo, o formato da serifa de uma fonte não é
relevante para uma fonte monoespaçada ou sem serifa.

\section{Você vê desafios? Quais?}
\begin{enumerate}
	\item Determinar quais features utilizar. Como dito anteriormente, cada 
	caractere pode se transformar em uma feature distinta para determinar uma 
	fonte.
	\item Produzir um um conjunto de treinamento satisfatório para fornecer como
	entrada ao classificador.
\end{enumerate}

\section{Ousaria desenhar os passos para chegar à solução desejada? Quais seriam esses passos?}
A solução para o problema consiste primordialmente na escolha correta de quais
características são relevantes para uma boa classificação, e assim, evitando
o problema da ``maldição da dimensionalidade'' e em seguida a escolha
de um classificador que melhor se encaixe no problema proposto.


\end{document}

