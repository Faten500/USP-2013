%% ------------------------------------------------------------------------- %%
\chapter{A Graduação em Ciência da Computação}
\label{cap:a_graduacao}
Realizo neste capítulo um balanço sobre os cinco anos de graduação em Ciência
da Computação realizados em duas universidades bastante distintas: a 
Universidade Federal de Santa Catarina, por dois anos, e a Universidade de São
Paulo, por três anos. A primeira graduação, por estar localizada juntamente
com os cursos de engenharia da UFSC, tem foco bastante voltado à parte
tecnológica da Computação, com grupos fortes fazendo pesquisa em Banco de Dados,
Sistemas Operacionais e Engenharia de Software. Já a segunda, localizada dentro
do Instituto de Matemática e Estatística, foca muito mais nos estudos de 
Matemática e conteúdos de fundamentos de Computação como Algoritmos, Grafos e 
Autômatos. 

Felizmente, essas diferenças se tornam complementares quando há interesse 
do aluno de tentar absorver o que cada lugar tem de melhor. Me considero um
sortudo por ter a chance de abraçar essa oportunidade e espero ter aproveitado
ao máximo o que me foi oferecido. De fato, encerro esse ciclo com a sensação de
dever cumprido, seja em termos acadêmicos, concluindo a graduação com boas notas
e cumprindo com meus deveres estudantis, seja em termos sociais, cultivando 
boas amizades em cada lugar e aproveitando o que a Universidade oferece a seus
alunos: restaurante universitário, biblioteca, cursos de idiomas, esportes,
viagens acadêmicas, festas universitárias, iniciação científica etc.

\section{Disciplinas cursadas relevantes para o desenvolvimento do TCC}
\label{sec:disciplinas_relevantes}
Desenvolver este TCC foi uma amostra bastante relevante de como os conteúdos em
Ciência da Computação estão relacionados, em maior ou menor grau. Felizmente, 
a área de Redes de Computadores é bastante eclética nas fundações que utiliza
para gerar seus resultados. (E que resultados! A Internet é fruto dos esforços
de inúmeros pesquisadores em Redes e sempre gerou fascínio em mim. Compreender
seu funcionamento e a sua capacidade de mudança foram alguns dos motivos que 
me fizeram escolher Computação como a carreira que quero perseguir
profissionalmente.)

Nesta seção é apresentada uma lista de cursos que fizeram a diferença para o
desenvolvimento da minha monografia, direta ou indiretamente, em ordem
cronológica.

\subsection{Programação Orientada a Objetos II}
\begin{description}
	\item[Universidade] UFSC
	\item[Professor] Luiz Fernando Bier Melgarejo
\end{description}
POO II é uma disciplina obrigatória do segundo semestre de Computação na UFSC.
O propósito da matéria é cobrir os principais conceitos de Orientação a Objetos
e pô-los em prática em um projeto. O destaque dessa disciplina não é tanto o
conteúdo mas sim a metodologia de ensino do professor.

Após uma rápida introdução ao \emph{framework} a ser usado durante a disciplina, os
alunos começam desenvolvendo mini-projetos de interesse próprio e usando os
(Poucos) conceitos que conhecem de POO aprendidos em POO I. Durante o semestre é
obrigação do aluno se oferecer para apresentar os códigos que vem produzindo 
no(s) projeto(s). É nesse momento que entra o professor, criticando ferozmente o
trabalho do aluno, comentando brevemente sobre práticas de programação e padrões
de projeto que o aluno não conhecia mas que poderiam ser utilizados. É
importante notar que o professor nunca dava a solução do problema. Cabia aos
alunos mudar seus códigos por conta própria, pesquisar boas soluções e 
discutir com colegas para chegar a um consenso do que deve ser apresentado como
``solução'' em uma aula posterior.

O curioso é que os alunos que sobreviviam a tal provação eram na grande maioria
aprovados. Mas mais que apenas uma nota no histórico escolar, os alunos saiam
muito mais unidos que antes graças à necessidade de companheirismo para
enfrentar o ``temido'' professor. Além disso, absorviam muito mais conceitos
do que numa aula expositiva simples, afinal, ninguém queria ser criticado na 
frente dos seus colegas então todos pensavam muito em seus códigos antes de 
fazer a apresentação.

Para este TCC, foi necessário analisar bastante código dos simuladores
utilizados além de empregar técnicas vistas nessa disciplina para o
desenvolvimento do PowerWorfkflowSim. Foi, também, nessa disciplina que criei
a maior intimidade com programação Java, muito utilizada neste trabalho.

\subsection{Organização de Computadores I}
\begin{description}
	\item[Universidade] UFSC
	\item[Professor] Luiz Cláudio Villar dos Santos
\end{description}

Organização de Computadores é uma disciplina obrigatória do terceiro semestre de
Computação na UFSC. O propósito da disciplina é apresentar a interface 
hardware-software. O destaque novamente vai para o professor.
Por um lado, ele é respeitado pelo seu profundo conhecimento de Computação em
geral e vivência profissional enquanto que por outro é temido pelo seu alto
índice de reprovação.

O professor faz questão de ministrar seu curso tal como fazem as melhores
universidades do país. O que pode ser problemático em termos de notas, 
novamente garante que os alunos que saiam da disciplina estejam preparados para
enfrentar problemas com muito mais confiança que em outros lugares.

Para o trabalho de conclusão, foram fundamentais a conceituação de consumo
energético e os \emph{tradeoffs} envolvidos na questão velocidade $\times$ 
economia energética. 

\subsection{Algoritmos em Grafos}
\begin{description}
	\item[Universidade] USP
	\item[Professor] Arnaldo Mandel
\end{description}

Algoritmos em Grafos é uma disciplina obrigatória do quinto semestre de
Computação do IME-USP. Aqui, são estudadas as formas de representação
computacional de grafos, um pouco de modelagem de problemas usando grafos e 
diversos algoritmos fundamentais da área.

Grafos e seus algoritmos são uma verdadeira cornucópia de utilidades em Ciência
da Computação. Uma estrutura simples e elegante é capaz de modelar e resolver
inúmeros problemas e de uma maneira normalmente intuitiva ou palpável. Para
um trabalho envolvendo Redes isso não pode ser diferente. Processei DAGs de
fluxos de trabalho científicos, apliquei uma busca em profundidade para 
gerar uma ordenação topológica com o objetivo de executar códigos em diversos nós 
interligados. Claramente, Grafos foram fundamentais para esse trabalho.

\subsection{Programação para Redes de Computadores}
\begin{description}
	\item[Universidade] USP
	\item[Professor] Daniel Macêdo Batista
\end{description}

Programação para Redes de Computadores é uma disciplina optativa do curso de
Computação do IME-USP. Aqui é feita uma abordagem \emph{top-down} do modelo
TCP-IP (Internet) de Redes de Computadores. São estudados os conceitos
fundamentais de cada camada, protocolos relevantes e realizados experimentos
para reforçar os conceitos aprendidos.

Apesar de já ter cursado Redes de Computadores na UFSC, fazendo um estudo
\emph{bottom-up} e estudando redes OSI juntamente com TCP-IP, sentia que meus 
conhecimentos na área estavam ainda fracos e cursar Programação para Redes de
Computadores foi uma ótima decisão. Por ser uma disciplina conjunta com a pós
graduação há alunos muito capacitados e interessados no estudo da disciplina.
Ainda, o professor consegue em seu curso um feito muito difícil em disciplinas de
graduação: motivar os alunos a discutir os assuntos em aula, enriquecendo-a.
Por fim, o professor mostra que tem conhecimento prático de programação em redes
ao programar e simular os conceitos vistos diretamente na aula.

Para o TCC o professor Daniel aceitou a tarefa de ser meu orientador, muito 
profissional, trouxe sempre materiais que tornassem o meu trabalho melhor e 
contribuiu com comentários muito pertinentes.


\section{Próximos Passos}
\label{sec:proximos_passos}
Com o fim da graduação iminente, surgem as dúvidas de qual carreira seguir.
Como comentado no começo do capítulo, considero que aproveitei a universidade
nas várias formas possíveis. Na área de pesquisa, participei de duas iniciações
científicas diferentes, uma na área de hardware na UFSC e outra na área de
ensino de Computação na USP. Assim, considero que este é um bom momento para 
gerar uma mudança na minha vida profissional, ingressando na indústria de
software.

Pretendo continuar trabalhando com Redes e Computação, mas agora em um
contexto mais prático. Uma das coisas que mais sinto falta na carreira acadêmica
é ver alguma ideia desenvolvida ser utilizada rotineiramente por milhares
(Milhões? Bilhões?) de pessoas. Ou seja, ir além do \emph{paper}, da dissertação
e da tese. De qualquer forma, mantenho meu carinho especial pelos anos vividos
na universidade, de onde levo conhecimento, contatos e amigos para o resto da
vida.


