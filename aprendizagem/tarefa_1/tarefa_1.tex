\documentclass[brazil]{article}
\usepackage{graphicx}
\usepackage[T1]{fontenc}
\usepackage[utf8]{inputenc}
\usepackage{lmodern}
\usepackage{babel}
\usepackage{url}


\begin{document}


\title{Tarefa 1: Descrever um problema}
\author{Pedro Paulo Vezzá Campos - 7538743}
\date{\today}

\maketitle

%\begin{abstract}
%\end{abstract}


\section{Definição do problema}
Sou o atual mantenedor do sistema MatrUSP (\url{http//bcc.ime.usp.br/matrusp}), 
um combinador de matrícula para alunos de graduação da USP. Para o seu funcionamento
é necessário extrair diversas informações sobre disciplinas oferecidas a alunos de graduação, principalmente
do JupiterWeb mas também de outros lugares relevantes aos alunos como por exemplo
do CEPE ou cursos de idiomas da FFLCH, CAVC Idiomas, Poliglota, etc. Atualmente um script
acessa esses sistemas e em seguida, faz o \emph{parse} das informações disponíveis, 
estruturando os dados disponíveis em um banco de dados simples.

O sistema não possui vínculo oficial com a Universidade ou outras entidades 
e portanto não possui nenhum acesso privilegiado aos bancos de dados. Assim, é
necessário extrair a informação das páginas web de cada um dos serviços. Com isso
surgem diversos problemas como inconsistências das informações disponíveis e a
fragilidade do programa frente a mudanças radicais no design das páginas.


\section{Caracterização dos dados (como representar?)}
Imagino que a representação de cada uma das páginas seria próxima ao formato
\emph{bag of words} onde cada \emph{word} seria um tipo de dado procurado: String
para título da matéria e nome do professor; Data para período de início ou 
término da matéria; Hora para marcar o início e término de uma aula etc.


\section{O que/como seria uma solução (computacional) boa do seu problema?}
É muito importante que o sistema não gere falsos positivos, exibir matérias 
inexistentes ou com dados errados é muito prejudicial para a experiência do usuário, 
portanto, o sistema deve ter uma alta confiança na resposta dada.

O tempo de processamento não é um problema. O \emph{parsing} é feito em lotes, 
uma vez por dia. Qualquer tempo de execução de no máximo algumas horas é razoável.


\section{Quais informações de contexto podem ser úteis para o reconhecimento?}
\begin{itemize}
	\item Disciplinas de universidades tem duração de 4, 6 ou 12 meses.
	\item Informações como um dia da semana próximo de um horário é um forte indicativo
	que o conjunto represente um dia de aula.
	\item O nome de uma matéria normalmente está próximo a um código que a identifica.
\end{itemize}


\section{Você vê desafios? Quais?}
O principal desafio é a grande variabilidade de disposição das informações buscadas
em uma página. Cada site tem seu \emph{layout} e criar um algoritmo realmente 
robusto e que seja capaz de estruturar os dados disponíveis é um trabalho de
pesquisa atual tanto na academia quanto na indústria.

\section{Ousaria desenhar os passos para chegar à solução desejada? Quais seriam esses passos?}
Em princípio esse problema encaixa-se na área de processamento de linguagem natural,
mais especificamente no tópico de extração de informação. 

\begin{itemize}
	\item Criar um \emph{lexer} capaz de dividir o documento nos átomos relevantes
	ao problema descritos acima.
	\item Realizar um aprendizado supervisionado para permitir que o sistema infira
	relações entre os átomos que permita que ele deduza qual informação está sendo
	processada. 
	
	Em \cite{wikipedia} são citadas algumas técnicas recomendadas para a tarefa:
	
	\begin{itemize}
		\item Expressões regulares escritas manualmente
		\item Usando classificadores tais como o Naïve de Bayes ou modelos de entropia máxima
		\item Usando modelos sequenciais tais como cadeias de Markov ocultas ou
		campos aleatórios condicionais
	\end{itemize}
\end{itemize}

\section{Referências}
	\begin{thebibliography}{99}
	\bibitem{wikipedia}
	WIKIPEDIA, ``Information extraction'', Disponível em
	\url{http://en.wikipedia.org/w/index.php?title=Information_extraction&oldid=550994368}.
	Acesso em 14 de agosto de 2013.
	\end{thebibliography}

\section{Anexos}
\subsection{Uma página do JupiterWeb}
\begin{small}
\begin{verbatim}
   Universidade de Sao Paulo
   BRASIL

     * Graduac,ao
     * [1]Ajuda
     * [2]Guia USP Acessivel
     * [3]Matricula interativa
     * [4]Informac,oes Academicas
     * [5]Calendario USP
     * [6]Disciplinas
     * [7]Turmas



     * Acesso Restrito
     * [8]Entrar
     * [9]Esqueci a Senha
     * [10]Primeiro Acesso



   Disciplinas oferecidas



   [11][print_edit.gif]  Preparar para impressao


   [Logo_usp2.gif]
                  Jupiter - Sistema de Graduac,ao


             Instituto de Matematica e Estatistica

                     Ciencia da Computac,ao

   Disciplina: MAC0300 - Metodos Numericos da Algebra Linear


   [12]Clique para consultar os Requisitos desta Disciplina MAC0300

   Lista de Turmas oferecidas


   Codigo da Turma: 2013245
            Inicio: Aug 1 2013
               Fim: Dec 10 2013
     Tipo da Turma: Teorica
      Observac,oes:

   Horario             Prof(a).
   qua     10:00 11:40 Leonidas de Oliveira Brandao
   sex     08:00 09:40 Leonidas de Oliveira Brandao


                                        Vagas Inscritos Pendentes Matriculados
   Obrigatoria                             80        49         1           43
           IME - Ciencia da Computac,ao    80        44         1           43
   Optativa Livre                           0         1         0            0
   Alunos Especiais                         2         0         -            0
   Extracurricular                          0         1         0            0
        ______________________________________________________________________________________


   [13]Clique para consultar as Informac,oes da Disciplina MAC0300
     ______________________________________________________________________________________

                                                 [14]Creditos | [15]Fale conosco
                  (c) 1999 - 2013 - Departamento de Informatica da Codage/USP

References

   Visible links
   1. https://uspdigital.usp.br/jupiterweb/jupAjuda.jsp?codmnu=2209
   2. https://uspdigital.usp.br/jupiterweb/grdGuiaUSPAcessivel.jsp?codmnu=2210
   3. https://uspdigital.usp.br/jupiterweb/grdMatriculainterativa.jsp?codmnu=2211
   4. https://uspdigital.usp.br/jupiterweb/grdInformacoesAcademicas.jsp?codmnu=2212
   5. https://uspdigital.usp.br/jupiterweb/jupCalendario.jsp?codmnu=2213
   6. https://uspdigital.usp.br/jupiterweb/jupDisciplinaBusca?tipo=D&codmnu=2214
   7. https://uspdigital.usp.br/jupiterweb/jupDisciplinaBusca?tipo=T&codmnu=2215
   8. https://uspdigital.usp.br/jupiterweb/webLogin.jsp
   9. https://uspdigital.usp.br/jupiterweb/esqueciSenha
  10. https://uspdigital.usp.br/jupiterweb/primeiroAcesso
  11. javascript:OpenWindowToPrint()
  12. https://uspdigital.usp.br/jupiterweb/listarCursosRequisitos?coddis=MAC0300
  13. https://uspdigital.usp.br/jupiterweb/obterDisciplina?sgldis=MAC0300
  14. https://uspdigital.usp.br/jupiterweb/creditos.jsp
  15. https://uspdigital.usp.br/jupiterweb/jupColegiadoEmailLista

   Hidden links:
  16. http://www.usp.br/
\end{verbatim}
\end{small}

\subsection{Uma página de um curso do CEPE}
\begin{small}
\begin{verbatim}
   [1][logo_usp.jpg]
   [2]Inicio

Information

Menu

     * [3]CEPEUSP
          + [4]Historico
          + [5]Infraestrutura
          + [6]Normas
          + [7]Como frequentar
     * [8]Cursos
          + [9]Comunidade USP
          + [10]Diferenciados
          + [11]Infanto-juvenil
          + [12]Publico Adulto
          + [13]Terceira Idade
     * [14]Eventos
     * [15]Nucleos
          + [16]LAtiS
          + [17]NUMES
          + [18]NUPSEA
          + [19]NURI
          + [20]PET

Modalidade - Tenis

Comunidade USP - Curso

     * [21]Cursos
          + [22]Comunidade USP
               o [23]Alongamento
               o [24]Badminton
               o [25]Basquetebol
               o [26]Boxe Educativo
               o [27]Caminhada / Alongamento
               o [28]Canoagem
               o [29]Capoeira
               o [30]Corrida
               o [31]Deep Running
               o [32]Exerc. Individual para iniciantes
               o [33]Exerc. Localizados
               o [34]Fitness
               o [35]Futebol
               o [36]Futebol mais de 40 anos
               o [37]Ginastica Olimpica
               o [38]Hidroginastica
               o [39]Judo
               o [40]Karate
               o [41]Mat Pilates
               o [42]Musculac,ao
               o [43]Natac,ao
               o [44]Orientac,ao Nutricional
               o [45]Preparac,ao Fisica
               o [46]Programa Emagrecimento
               o [47]Remo
               o [48]Soft Tenis
               o [49]Tenis
               o [50]Treinamento Funcional
               o [51]Voleibol
               o [52]Yoga

   [53]Inicio > [54]Cursos > [55]Comunidade USP

O que e

   O curso de iniciac,ao destina-se a pessoas interessadas em adquirir experiencia na
   modalidade. O aluno aprende os movimentos basicos do jogo: batida de direita ("forehand"),
   batida de esquerda ("backhand") e saque ("service"). Aprende tambem a movimentar-se na
   quadra, a utilizar o paredao, inicio da execuc,ao do voleio ("volley"). Serao dadas noc,oes
   de regras, arbitragem e estrategia de jogo.

   A turma de aperfeic,oamento tem por objetivo aprimorar as habilidades adquiridas na
   iniciac,ao (batida de direita, batida de esquerda, paralelas e cruzadas e saque).
   Aprende-se o voleio, o "lob", o "smash" e a movimentac,ao na quadra, levando `a
   participac,ao em jogos internos individuais e em duplas.

Horarios

           Sexo           Dias    Horarios   Nivel Professor Idade     Local     Vagas Distrib.Vagas
   Masculino e Feminino 3-a/5-a  07h00-08h00 II    Thales    18 +  Q.Tenis 4 a 9 20    6/8/2013
   Masculino e Feminino 3-a/5-a  08h00-09h00 II    Thales    18 +  Q.Tenis 4 a 9 20    6/8/2013
   Masculino e Feminino 3-a/5-a  11h00-12h00 II    Thales    18 +  Q.Tenis 4 a 9 20    6/8/2013
   Masculino e Feminino 3-a/5-a  12h00-13h00 I     Thales    18 +  Q.Tenis 4 a 9 20    6/8/2013
   Masculino e Feminino 2-a/4-a  16h30-18h00 I     Eduardo   18 +  Q.Tenis 6 e 7 12    5/8/2013
   Masculino e Feminino 2-a/4-a  18h00-19h30 I     Eduardo   18 +  Q.Tenis 6 e 7 12    5/8/2013
   Masculino e Feminino 3-a/5-a  16h30-18h00 I     Eduardo   18 +  Q.Tenis 6 e 7 12    6/8/2013
   Masculino e Feminino 3-a/5-a  18h00-19h30 I     Eduardo   18 +  Q.Tenis 6 e 7 12    6/8/2013

   [56]Inicio | [57]Horario | [58]Localizac,ao | [59]Contato

   (c) 2009 - 2013 Universidade de Sao Paulo - Todos os direitos reservados.
   Prac,a 02, Prof. Rubiao Meira, 61 - Cidade Universitaria, Sao Paulo, SP - CEP 05508-110

References

   1. http://www.usp.br/
   2. http://www.cepe.usp.br/site/
   3. http://www.cepe.usp.br/site/?q=cepeusp
   4. http://www.cepe.usp.br/site/?q=cepeusp/historico
   5. http://www.cepe.usp.br/site/?q=cepeusp/infraestrutura
   6. http://www.cepe.usp.br/site/?q=cepeusp/normas
   7. http://www.cepe.usp.br/site/?q=cepeusp
   8. http://www.cepe.usp.br/site/?q=cursos
   9. http://www.cepe.usp.br/site/?q=cursos/comunidade-usp/
  10. http://www.cepe.usp.br/site/?q=cursos/diferenciados
  11. http://www.cepe.usp.br/site/?q=cursos/infanto-juvenil
  12. http://www.cepe.usp.br/site/?q=cursos/publico-adulto
  13. http://www.cepe.usp.br/site/?q=cursos/terceira-idade
  14. http://www.cepe.usp.br/site/?q=eventos
  15. http://www.cepe.usp.br/site/?q=nucleos
  16. http://www.cepe.usp.br/site/?q=nucleos/latis
  17. http://www.cepe.usp.br/site/?q=nucleos/numes
  18. http://www.cepe.usp.br/site/?q=nucleos/nupsea
  19. http://www.cepe.usp.br/site/?q=nucleos/nuri
  20. http://www.cepe.usp.br/site/?q=nucleos/pet
  21. http://www.cepe.usp.br/site/?q=cursos
  22. http://www.cepe.usp.br/site/?q=cursos/comunidade-usp
  23. http://www.cepe.usp.br/site/?q=cursos/comunidade-usp/alongamento
  24. http://www.cepe.usp.br/site/?q=cursos/comunidade-usp/badminton
  25. http://www.cepe.usp.br/site/?q=cursos/comunidade-usp/basquetebol
  26. http://www.cepe.usp.br/site/?q=cursos/comunidade-usp/boxe-educativo
  27. http://www.cepe.usp.br/site/?q=cursos/comunidade-usp/caminhada-/-alongamento
  28. http://www.cepe.usp.br/site/?q=cursos/comunidade-usp/canoagem
  29. http://www.cepe.usp.br/site/?q=cursos/comunidade-usp/capoeira
  30. http://www.cepe.usp.br/site/?q=cursos/comunidade-usp/corrida
  31. http://www.cepe.usp.br/site/?q=cursos/comunidade-usp/deep-running
  32. http://www.cepe.usp.br/site/?q=cursos/comunidade-usp/exerc-individual-para-iniciantes
  33. http://www.cepe.usp.br/site/?q=cursos/comunidade-usp/exerc-localizados
  34. http://www.cepe.usp.br/site/?q=cursos/comunidade-usp/fitness
  35. http://www.cepe.usp.br/site/?q=cursos/comunidade-usp/futebol
  36. http://www.cepe.usp.br/site/?q=cursos/comunidade-usp/futebol-mais-de-40-anos
  37. http://www.cepe.usp.br/site/?q=cursos/comunidade-usp/ginastica-olimpica
  38. http://www.cepe.usp.br/site/?q=cursos/comunidade-usp/hidroginastica
  39. http://www.cepe.usp.br/site/?q=cursos/comunidade-usp/judo
  40. http://www.cepe.usp.br/site/?q=cursos/comunidade-usp/karate
  41. http://www.cepe.usp.br/site/?q=cursos/comunidade-usp/mat-pilates
  42. http://www.cepe.usp.br/site/?q=cursos/comunidade-usp/musculacao
  43. http://www.cepe.usp.br/site/?q=cursos/comunidade-usp/natacao
  44. http://www.cepe.usp.br/site/?q=cursos/comunidade-usp/orientacao-nutricional
  45. http://www.cepe.usp.br/site/?q=cursos/comunidade-usp/preparacao-fisica
  46. http://www.cepe.usp.br/site/?q=cursos/comunidade-usp/programa-emagrecimento
  47. http://www.cepe.usp.br/site/?q=cursos/comunidade-usp/remo
  48. http://www.cepe.usp.br/site/?q=cursos/comunidade-usp/soft-tenis
  49. http://www.cepe.usp.br/site/?q=cursos/comunidade-usp/tenis
  50. http://www.cepe.usp.br/site/?q=cursos/comunidade-usp/treinamento-funcional
  51. http://www.cepe.usp.br/site/?q=cursos/comunidade-usp/voleibol
  52. http://www.cepe.usp.br/site/?q=cursos/comunidade-usp/yoga
  53. http://www.cepe.usp.br/site/
  54. http://www.cepe.usp.br/site/?q=cursos
  55. http://www.cepe.usp.br/site/?q=cursos/comunidade-usp
  56. http://www.cepe.usp.br/site/?q=inicio
  57. http://www.cepe.usp.br/site/?q=horario
  58. http://www.cepe.usp.br/site/?q=localizacao
  59. http://www.cepe.usp.br/site/?q=contato
\end{verbatim}
\end{small}
\end{document}

